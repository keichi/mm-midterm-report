\section{クラウドプロバイダ従事者の不正}

\subsection{概要}
クラウドプロバイダ従事者による悪意の行動は,全データの機密性,完全性,
可用性,IP およびすべてのサービスに影響を与える可能性があるため,
企業の評判,顧客の信頼,従業員の忠誠心と経験,知的財産そして個人の秘密データ
などに大きな影響を与えると考えられる.
よって,クラウドプロバイダ従事者による悪意の行動に対する対策を
講じることは重要である.


\subsection{事例}

\subsubsection{宇治市住民票データ流出事件}
文献\cite{uzi}によると,A市は,住民基本台帳のデータを使用して乳幼児の検診システムを開発すること
を企画し,開発業務をB社に委託した.業務委託契約書では,データの複写・
複製の禁止,市長の書面による承諾がある場合を除き再委託は禁止されていた.
ところが,すでにC社がシステム開発を手がけていたところから,
B社はA市の承諾を得てC社に再委託した.さて,C社は,A市に知ら
れぬようにさらにD社に業務全体を再委託した.A市との打ち合わせはD社の
代表者とその社員Eが行ないましたが,C社の名刺を使いC社の社員であることを
装っていた.そして,Eとアルバイトの大学院生TがA市庁舎に出向き開発
業務を行なった.EとTの両名は,市庁舎内で作業を行なったが,
作業が遅れたため,データを持ち帰って作業を進めたいと市の担当職員
に了解を求め,職員の承諾を得た上,データをコピーして持ち帰り,D社の社内
で作業をするようになった.しかし,あろうことか,Tは,D社内でデータ
を自己のコンピュータにコピーし,これをMO(光磁気ディスク)にコピーした
うえ,名簿業者F社に25万8000円で売却した.F社に流出したデータは,
住民記録18万5800件,外国人登録関係3297件,法人関係の2万8520件,合
計21万7617件であった.F社はさらに,この情報を結婚相談事業者に21万7608件,婚礼衣
装業者に1324件,別の名簿業者G社を通じ251件のデータを販売した.新聞
報道により情報流出を知ったA市は,直ちに関係者に連絡をとり,F社,結婚相
談業者,婚姻衣装業者に対しデータを消却してもらい,MO等のデータを回収し
た.しかし,G社とは連絡がとれなかった.A市は,市政だよりなどで
以上の事実を市民に説明の上,謝罪し,再発防止策を講じるとともに,Tを刑事
告発した.以上の事実に関し,A市の住民Xら3名は,本件情報に含まれる
個人情報が第三者に販売され,またホームページ上で誰でも購入することが出
来る状況におかれたことによってプライバシー権が侵害されたとして,A市に対
し,損害賠償請求訴訟を起こした結果,
第1審の裁判所(大阪地裁)も第2審の裁判所(大阪高裁)もA市の損害
賠償義務を認めた.

\subsubsection{ベネッセ顧客情報流出事件}
2014年7月9日に発覚したベネッセコーポレーション(ベネッセ)の大規模個人
情報流出事件である.流出した顧客情報が最大で2070万件に及ぶ大規模なものである.
流出した情報は,進研ゼミなどといったサービスの顧客の情報で,子供や
保護者の氏名,住所,電話番号,性別,生年月日をはじめ,クレジットカード番号
までも流出した可能性がある.ベネッセ側は,社内調査により,
データベースの顧客情報が外部に持ち出されたことから流出したことが判明した.



\subsection{暗号化状態処理技術}
本節では,クラウドプロバイダ従事者の不正の対策として,暗号化状態処理技術を紹介する.
暗号化状態処理技術とは,データを暗号化したまま処理が可能な暗号技術の総称である.
文献\cite{sym}によると,具体的な暗号化状態処理技術として,次のものが挙げられる.

\begin{itemize}
\item 秘匿検索 … データを暗号化したまま検索
\item 再暗号化 … 元の暗号文を復号することなく,別の鍵で復号できるようにする変換
\item 秘匿計算 … 暗号化したデータ同士の演算.
\end{itemize}

現在,暗号化されたデータはストレージやネットワーク帯域を節約することが
出来ないため(暗号化されたデータは乱数列と見分けが付かないため,圧縮・重
複除去することができない),サービスプロバイダはデータを暗号化したまま保
存するのを避ける傾向がある.また,暗号化されている機密情報がデータベー
スに格納されている場合でもアプリケーション側からの処理指示のたびにデー
タベース上で暗号化されたデータが復号されるため,データベースの管理者や
管理者の権限を盗み取った者によるデータ窃取のリスクがある.このようにデー
タを扱いやすさと機密性を両立の困難さがある.そこで,暗号化状態処理技術
を用いることによってデータの扱いやすさと機密性を両立できるという利点が
ある.この暗号化状態処理技術は,既に製品化も始まっている.これらの技術
の各社の実用化動向は表\ref{tab:real} のとおりである.

\begin{table}[htbp]
  \begin{center}
    \small{
    \begin{tabular}{llll}
      \hline
      処理の種類 & 企業 & 主な用途 & 実用性\\
      \hline
      & 日立制作所 & ゲノムデータ分析 & 1万件規模のデータ検索が約8ミリ秒で可能 \cite{hitachi}\\
      秘匿検索 & 三菱電機 & データベース検索 & 10万件規模のデータ検索が約1〜3秒で可能 \cite{mitubisi}\\
      & 富士通研 & ビッグデータ解析 & 1万6千文字の暗号化データの全文検索が約1秒で可能\cite{fujitsu}\\
      \hline
      再暗号化 & 東芝 & 情報共有 & 1MBの暗号化データの再暗号化が85ミリ秒で可能 \\
      & NICT & 情報共有 & 128バイトの暗号化データの再暗号化が約35~40ミリ秒で可能\cite{nict}\\
      \hline
      秘匿計算 & 富士通研 & 生体認証 & 生体情報の照合が約5ミリ秒で可能\cite{fujitsu}\\
      & 日立製作所 & ビックデータ分析 & 10万件規模のデータに対する相関ルール分析が約10分で可能 \cite{hitachi}\\
      \hline
    \end{tabular}
    \label{tab:real}
    \caption{各社の暗号化状態処理の実用化動向}
  }
  \end{center}
\end{table}


\subsection{秘密分散法}
\subsubsection{概要}
クラウド・コンピューティングは,低コストやサービスの早期立ち上げ等に強
い期待がある反面,データ管理が不明確で,とりわけセキュリティ面での不安
が導入の最大の阻害要因となっている.また,先の東日本大震災で,
データのバックアップを遠隔地に保管するニーズが急速に高まっているなか,
安価なパブリックラウドサービスでは,相変わらずセキュリティの不安が解消
できていないがために手が打てずにいるという現状がある.
そこで,本節では,秘密分散法を紹介する.秘
密分散法は安全にデータの管理をするセキュリティ技術の一つである.秘密に
しておきたいファイルなどの情報を他人に知られないようにするためには,共
通鍵暗号化方式を用いて暗号化するのが一般的である.また,認証,機密通信
などではPKI(Public Key Infrastructure)が利用されている.しかし,鍵を使
う方式は,秘密鍵の管理,証明書の発行・更新など利用に際し,運用コストが
かかるという問題がある.しかし,秘密分散法は「秘密」にすべき情報を複数
の「分散情報」に分け,それらがある決めた数集まらないと元のデータを復元
することができないため,これらの鍵暗号方式に比べ,運用が容易,セキュリ
ティ強度が高いと期待されている.

\subsubsection{秘密分散法の特徴}
秘密分散法では,秘密にしたい情報を複数の「分散情報」に分ける.
共通鍵暗号などの暗号化手法を用いた場合,データになんらかの暗号化のため
の変換を加えて,元データの復元を困難にしているが,データそのものが内包
されているため,コンピュータにおけるCPUの処理スピードの向上,暗号解読技
術の進歩などにより解読される可能性が残る.しかし,秘密分散法の場合は,
データそのものが複数に分かれるため,たとえ,分散情報の1つを取得したとし
ても,その情報から,元データを復元することはできない.このことから,高
い安全性が得られる.鍵を使った暗号化と比較した場合,鍵の管理がないため,
保管に関する運用の手間を削減,鍵の漏洩による暗号解読リスクがないという
メリットもある\cite{sdsm}.

\subsubsection{秘密分散法の応用}
実際に秘密分散法を用いたセキュアな分散ストレージシステムは既に各社により,開発されている.
例えば,NRIセキュアテクノロジーズと日本マイクロソフトが連携して提供している
サービスである世界分散ストレージサービス\cite{nrimicrosoft}や,
NTTソフトフェアが提供するTrustBind/Hybrid Storage\cite{ntt}
などが既に実用化されている.
