\chapter{序章}

\section{社会的背景}
情報化社会の発達に伴い,現実社会で盗難や犯罪が行われるように,情報化社会でも同様に盗難,犯罪の危険性がもたらされる.さらにインターネットの普及により,個人や企業を問わず,クラッキングによるコンピュータへの侵入や情報の盗難,データの改ざんといった様々な攻撃を食らう可能性がますます大きくなった.また,最近では,企業や組織が保有している個人情報などのデータが外部へ漏えいしてしまうという事件が多発している.そのため,セキュリティ対策というものが重要視されている.企業や組織は,どのようなリスクがあるか,十分にリスク分析し,技術的な対策を練る必要がある.

\section{技術的背景}
モバイル,クラウドといった革新的な技術の発展によって,人々はより容易にインターネットの世界にアクセスし,サービスを利用・提供できるようになった.
しかし,悪意のある第三者の視点から見た場合,これらの技術の大量に集約されたリソースとデータは,攻撃者にとっても格好の標的となる.
そのため,クラウドストレージなどのインターネットの様々なサービスや無線LANアクセスポイントに代表されるインターネット基盤においてもセキュリティ問題の重要性が高まっている.\\
まず,クラウドコンピューティングにおいては,特定ベンダの独自技術に大きく依存したサービスを利用した場合に,他ベンダの同種のサービスへの乗り換えが困難になるロックインの問題や,クラウド従事者自身が悪意のある行動を取れないようにして,クラウドサービスプロバイダの信頼性を得るための仕組みが必要となる.
そして,クラウド環境では計算能力やストレージ,ネットワークが共有されるといった点から,サイドチャネル攻撃などの問題を防ぐための対策も必要となる.
それに加え,クラウド側だけでなく,クラウド-ユーザ間のデータ転送途上における攻撃にも対応しなくてはならない.
また,社会的な側面の問題として,クラウドのデータセンターが保存される国ごとの司法権の違いからくるリスクを考慮することも重要な観点となる.これらのクラウドコンピューティングにおける情報セキュリティに関する利点やリスクについて,ENISA(欧州ネットワーク情報セキュリティ庁)がドキュメント\cite{cloudrisk}を発行している.\\
クラウドコンピューティングの他にも,爆発的な普及により現在至るところで設置・利用されている無線LANアクセスポイントに関しても,誰にでも容易に通信ができるという点で,多くの考慮すべきセキュリティの問題がある.
また,どのようなインターネット上のサービスにおいても,ネットワークを介しそのシステムのサーバ本体の脆弱性が直接攻撃されることが考えられ,管理者はシステムを正しく理解し,そのような攻撃からシステムを守る必要がある.

\section{目的}
本演習では,実際的なセキュリティリスクについて実習および調査を行い,それらのリスクに対抗する方案を考察した.
これにより,実務的なセキュリティ知識を習得して自らのセキュリティ実践力を高めることを目的とした.
実習では,京都大学,奈良先端科学技術大学院大学,北陸先端科学技術大学院大学,東京大学といった他大学の学生と協同して進め,分野,地域を超えてセキュリティリスクに取り組んだ.
取り上げたセキュリティリスクは,クラウドシステムに関するセキュリティ,最新の脆弱性,無線LANに対する攻撃,システムへの攻撃,システムの侵入及び解析である.

クラウドシステムの普及に従い,クラウドセキュリティの重要性もまた高まっている.
本報告書では,クラウドシステムの代表的なリスクとしてロックイン,クラウドプロバイダ従事者の不正,データ転送途上における攻撃,隔離の失敗,司法権の違いから来るリスクを取り上げた.
各リスクについて,概要と具体的な事例を述べ,社会的,管理的,技術的にどのような対策を採ることができるか考察した.

また,今年注目された最新の脆弱性について調査を行い,攻撃方法や対策について考察した.
具体的には,Heartbleed,Shellshock,POODLEの3つの脆弱性について調べた。

そして,無線LANに対する攻撃,システムへの攻撃,システムの侵入及び解析のそれぞれについて,行った実習の概要を述べ,実習の結果を基にどのような対策を採れるか考察した.
