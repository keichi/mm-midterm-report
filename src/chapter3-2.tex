\section{Shellshock}

\subsection{概要}

Shellshockは,bash (Bourne-Again SHell)
というシェルにおける一連の脆弱性の名称である.\cite{uscert}
この脆弱性が存在するシステムにおいては,攻撃者が任意のコマンドを実行できてしまう.bashは
多くのUNIXベースの環境においてデフォルトのシェルとして使用されているため,非常に広範囲に
渡って影響を与える脆弱性である.また,bashを用いて間接的にコマンドを実行するプログラム
(例:Apacheのmod\_cgi・OpenSSHのsshd・DHCPクライアント・qmailなど)
も影響を受けるため,攻撃者はリモート環境から攻撃することも可能になる.

\subsection{脆弱性の内容}

Shellsock脆弱性の原因は,bashの環境変数の処理におけるバグである.\cite{cve20146271}
bashの機能の1つとして,よく使う処理を関数として定義し,後ほど再利用する機能がある.

\begin{verbatim}
function hello {
 echo "Hello, world!"
}
hello
\end{verbatim}

上記のコードでは\texttt{hello}という関数を定義し,次に呼び出している.また,bashでは環境
変数を用いて定義することもできる.

\begin{verbatim}
env hello="() { echo 'Hello, world!'; }" bash -c hello
\end{verbatim}

上記のコードのように,\texttt{() \{}で始まる文字列を環境変数に設定すると,新しく開いたシェル
で関数定義が暗黙的にインポートされ,使用できるようになる.この挙動もbashの仕様の範囲
内である.Shellshock脆弱性の原因となったのは,このような環境変数での関数定義の
後ろに続くコードまで実行してしまっていたことである.

\begin{verbatim}
env x="() { :;}; echo 'vulnerable'" bash -c "echo this is a test"
\end{verbatim}

上記の例では環境変数\texttt{x}に空の関数定義を設定しているが,その際に関数定義の後ろの
\texttt{echo "vulnerable"}まで実行してしまう問題があった.シェルを開くたびに関数定義
の後ろのコードが実行されるため,攻撃者はこれを利用することで,任意のコードを実行
できてしまう.
他にもbashのパーサには様々なバグが存在し,これらを利用することで任意のコードの実行や
DoS攻撃が可能になってしまっていた.\cite{cve20147169, cve20147186, cve20147187, cve20146277, cve20146278}

また,間接的にbashを実行しているアプリケーションもこの脆弱性の対象となった.例えば,
Apacheで動的コンテンツを配信するためのモジュールであるmod\_cgiは,WebサーバからCGI
スクリプトに対してデータを渡すために,シェルの環境変数を利用していた.よって,クライアント
から悪意のあるHTTPリクエストを送信することで,Shellshock脆弱性をつくことができる.

\begin{verbatim}
GET / HTTP/1.1
User-Agent: () { :;}; echo vulnerable
\end{verbatim}

上記の例では,\texttt{User-Agent}に攻撃コードが設定されている.mod\_cgiでは\texttt{User-Agent}の
値は環境変数に設定されるので,その際に\texttt{echo vulnerable}が実行されてしまう.
同様の攻撃がOpenSSHのsshd,DHCPクライアント,qmailなどについても可能である.

\subsection{対策}

Shellshock脆弱性に対する最も直接的かつ効果的な対策は,bashのアップデートである.脆弱性を
防ぐパッチを当てたバージョンのbashが配布されているので,これで利用しているbashの
バイナリを置き換えることによって対策を行える.また,主要なLinuxディストリビューション
からはパッチしたbashをインストールするパッケージが提供されているので,より簡単に
脆弱性対策が可能である.関数自動インポート機能自体がセキュリティリスクであるとし,
デフォルトではbashの関数自動インポート機能を無効にしているディストリビューションもある.\cite{freebsd}
