\chapter{システム侵入解析演習}

\section{ARP (Address Resolution Protocol) キャッシュポイズニング}
\subsection{概要}
現在、ネットワーク上に存在するコンピュータに対して様々な攻撃が考えられるが、その中でもARP\ Cache\ Poisoning
と呼ばれる攻撃がある。ARP\ Cache\ Poisoningとは、多くのOSにおいて不可避な攻撃であるが、
この要因としては、ARP\ Cache\ PoisoningがEthernetの仕様に基づく攻撃である点があげられる。この攻撃はコンピュータ間の通信に攻撃者が割り込み、通信相手のコンピュータを偽って正しい通信として認識させる。
攻撃者が対象間に割り込んでいるため、コンピュータ間のすべての通信に影響を与えることが可能となってしまう。これにより情報の漏洩や改竄だけでなく、他の攻撃へつながってしまう危険が懸念される\cite{arp}。

\subsection{ARPとは}
既存のEthernet環境においては、ネットワーク層のIPアドレスだけでは通信を行うことが出来ないので、通信を行うためにはMACアドレスを取得する必要がある。
そこで、ARPはIPアドレスとMACアドレスの関連付けを行っており、対応する各ホストのARPテーブルはキャッシュとしてOSが保持している。\cite{ip-arp}
\subsection{問題点}
ARPにはセキュリティ上の致命的な問題が2つ存在している。
1つ目は受信したARPパケットが正しい送信者から送信されたパケットが本当に正しいパケットか判断することができない点である。ARPは全てのパケットは
必ず正しい送信者が送信したとするので、攻撃者が正しい送信者と偽装して
いた場合も見破ることができない。2 つ目は、受け取ったARP パケットの内容を検閲せずにOS
がキャッシュとして保持するので、不正なARP パケットもそのまま受理してしまう可能性がある
点である。ARP Cache Poisoning は以上のARP の仕様を悪用した攻撃手法である。

\subsection{対策}
ARP Cache Poisoning に関する対策としては、
\begin{itemize}
\item ARP テーブルをStatic に登録する\\
偽造ARP 応答によるARP テーブルの更新をStatic 登録により防ぐ。
\item DHCP スヌーピング機能を適用\\
ARP セキュリティーオプションを利用することで、不正なARP パケットを破棄することが
可能
\item IEEE 802.1X 認証+クライアントPC 管理\\
パスワード認証(MD-5、PEAP、TTLS)での持ち込みPC 制限は困難なため、IEEE 802.1X
(TLS 認証)により不正端末の持ち込みを防止する。サプリカントソフトによってはパスワー
ドの事前定義が可能なので有効な場合もある。
\item クライアントPC 管理により攻撃ツールのインストールを防止\\
攻撃用ツールのインストールなしに攻撃を行うことは困難なので、グループポリシーの適用
やAdmin 権限の削除などを行う
\end{itemize}
が挙げられる。\cite{arp-taisaku}他には、ダイナミックARP 検査の動作があり、これは、ダイナミックARP 検査
により、有効なARP Request およびARP Reply のみがスイッチ上で転送されることが保証され
ることを用いる。具体的には、ダイナミックARP 検査が有効にされたVLAN 上では、Catalyst
スイッチは以下のアクティビティを実行し、
\begin{enumerate}
\item 信頼できないポート( no ip arp inspection trust ) 上のすべての[ ARP Request ] および[
ARP Reply ] を代行受信する。
\item 代行受信された各パケットに有効なIP アドレスとMAC アドレスのバインディングがある
かどうかを確認。
\item Catalyst 上で保持しているバインディングデータベース上に存在しない組み合わせである場
合、その無効なARP を破棄。
\end{enumerate}
という動作を行う。

\section{DNS (Domain Name System) キャッシュポイズニング}
\subsection{概要}
DNS (Domain Name System)とは,ホスト名とIPアドレスとを紐づける情報を提供する,階層的な分散型データベースシステムである.
インターネット上のアプリケーションの多くはDNSを前提として設計されており,DNSはインターネットの基盤サービスとも言われている.
DNSサーバは,クライアントからホスト名を受け取り,対応するIPアドレスを返すサービスや,メールアドレスに含まれるドメイン名から,利用するメールサーバ名を返すといったサービスを提供する.

DNSサーバには,ドメインの原本情報を管理するコンテンツサーバと,クライアントとコンテンツサーバの中間にあり,クライアントに代わってコンテンツサーバに問い合わせる再帰動作を行うキャッシュサーバがある.キャッシュサーバでは,コンテンツサーバへの問い合わせ結果を一時的に記憶しておく.あるクライアントからの問い合わせに対する結果を記憶している場合は,コンテンツサーバへの問い合わせは行わず自らが記憶している結果を回答として返す.

DNSキャッシュポイズニングとは,DNSサービスを提供しているサーバに偽の情報を覚えこませ汚染する攻撃手法であり,主に汚染されたサーバからDNSサービスを受けるユーザを悪意のあるWebページへと誘導するために用いられる.

また,より効率良くDNSキャッシュポイズニングを行う方法として,Kaminsky Attackが挙げられる.

\subsection{攻撃手法}
ここでは、一般的なDNSキャッシュポイズニングの方法と、Kaminsky Attackについて述べる。

\subsubsection{一般的なDNSキャッシュポイズニング}
DNSキャッシュポイズニングを実現する一般的な手法は次のようなものがある.攻撃者がキャッシュサーバに偽の問い合わせをし,キャッシュサーバが再帰動作を行っている間に,本物のコンテンツサーバからの問い合わせよりも早く偽の問い合わせ回答を送り込むことで,キャッシュサーバは偽の問い合わせ結果をキャッシュとして記憶してしまい,汚染される.

DNSキャッシュポイズニングにより引き起こされる被害として考えられるものが,罠の張られた悪意あるWebページへと誘導されてしまう被害である.
クライアントがとあるWebページにアクセスしたいとき,DNSキャッシュサーバへと問い合わせを行う.
もし,このキャッシュサーバが既に汚染されており,クライアントがアクセスしたいWebページに関しての偽の情報が記憶されていたら,クライアントは悪意ある偽のWebページへと簡単に誘導されてしまう.悪意あるWebページにアクセスしてしまったクライアントは,フィッシング詐欺や電子メールの盗難などといった直接的かつ甚大な被害を被る可能性がある.

もう一つの被害として考えられるのは,DoS攻撃のための増幅装置として汚染されたキャッシュサーバが用いられる被害である.
キャッシュサーバを汚染し,回答データが異常に大きくなるような偽の情報を記憶させた上で,攻撃者が発信元を詐称し攻撃対象を装って問い合わせを行うと,キャッシュサーバが異常に大きな回答データを攻撃対象へと送信してしまう.理論的には49倍程度のトラフィック増幅が可能であると報告されている.\cite{ipa}

\subsubsection{Kaminsky Attack}
2008年にセキュリティ研究者のDan Kaminskyによって発表された新たなDNSキャッシュポイズニングの方法をKaminsky Attackと呼ぶ.\cite{kaminsky}通常のキャッシュポイズニングでは,キャッシュのTTLを十分に長くすることで成功率を低くできると考えられてきたが,Kaminsky AttackではTTLの長さに関係なく攻撃することが可能となり,その前提が崩れることとなった.\\
Kaminsky Attackの具体的な手順は以下の通りである.

\begin{enumerate}
	\item 攻撃対象のキャッシュサーバに対し,乗っ取りたいドメイン名と同じドメイン内で実在しないドメイン名を問い合わせる.例えばwww.example.jpのドメイン名を乗っ取る場合,○○○(任意の文字列).example.jpを問い合わせる.
	\item 問い合わせを依頼されたキャッシュサーバは,キャッシュにないためコンテンツサーバに問い合わせる.
	\item 攻撃者は偽の応答として参照先を示す内容のパケットを,ターゲットのキャッシュサーバに送り込む.例えば「そのドメイン名に関しては次のコンテンツサーバに問い合わせよ.www.example.jp IPアドレスはaa.bb.cc.dd」とする.ここでaa.bb.cc.ddは攻撃者が用意した偽のコンテンツサーバのIPアドレスである.
\end{enumerate}

この攻撃方法では,ランダムなドメイン名をキャッシュサーバに問い合わせることで,強制的に外部に問い合わせを行わさせることが可能であり,またIDの不一致などで攻撃に失敗しても,ドメイン名を変えてすぐに再チャレンジできることから,これまでのキャッシュポイズニングよりも効率的に攻撃を行うことができる.

\subsection{対策}
DNSキャッシュポイズニングが成功するまでには,
\begin{enumerate}
	\item 偽の応答を注入する
	\item 注入した偽の応答がキャッシュされる
	\item キャッシュされた偽の応答が実際に利用される
\end{enumerate}
という三つの関門がある.それぞれの関門に対し,
\begin{enumerate}
	\item 偽の応答を注入されない,若しくはされにくいようにする
	\item 受け取った応答を厳重にチェックし,うかつにキャッシュしないようにする
	\item 攻撃を検知し,偽情報を利用しないよう対応する
\end{enumerate}
といった対応をすることができる.\cite{jrps}

具体的には,ソースポートランダマイゼーションの実施,クライアントの限定とパケットフィルタリング,DNSSECの採用といった手法を採ることができる.

\subsubsection{ソースポートランダマイゼーション}
従来のキャッシュサーバの実装では,問い合わせに用いるUDPポートは1つに固定,あるいは決められた範囲に限られたポートしか使用しないものがほとんどである.そこで,使用するポートを問い合わせ毎に毎回変化させランダム化することで,IDのみがランダムでポートが固定の場合に比べ,キャッシュポイズニングが成功する確率を大幅に下げることができる.

\subsubsection{問い合わせDNSクライアントの限定やパケットフィルタリング}
先述した攻撃手法のリスクを軽減するためには,攻撃者が自由に攻撃できないようにすることが重要となるため,キャッシュサーバに問い合わせ可能なクライアントを限定する,ソースアドレスの偽装されたパケットを遮断することが有効であると考えられる.IDを総当たりで推測する手法では,異常なパケット(問い合わせに使用していないIDの応答や,大量のコンテンツサーバからの応答)が観測されるため,これらを検知し防御することが有効である.

\subsubsection{DNSSEC}
根本的な解決策として,DNSSECが挙げられる.DNSSECとは,インターネットの安全性を向上させるための,改良されたDNSプロトコルのセキュリティ拡張機能である.DNSSECでは,件にサーバによって応答に電子署名が行われ,キャッシュサーバがその署名を検証することで,応答の偽装・改竄を検知することができる.それによって,キャッシュポイズニング攻撃に対する防御はほぼ完全となる.しかし,DNSSECを導入するには,関連するサーバ全てをDNSSEC対応のものに更新する必要や,電子署名に用いる鍵の管理や配布手段の確立など,様々な課題があるため,DNSSECの普及促進をどのように進めていくかが問題となっている.

DNSSECは,DNS SECurity Extension(DNSセキュリティ拡張)の略称であり,公開鍵暗号を使って検索側が受け取ったDNSレコードの出自・完全性を検証できる仕組みのことである.\cite{dnssec}
DNSSECでは,DNSサーバ側がDNS応答に電子署名を付与しておく.これにより,問い合わせ側が署名を検証することで改竄の有無,出自の確かさ,完全性を確かめることができる.
DNSSEC以外にもDNSキャッシュポイズニングを防御するための技術は存在するが,標準化されているものはDNSSECのみである.\cite{RFC2535} \cite{RFC3110} \cite{RFC4033}

DNSSECの利用に当たっては,DNSSECを利用したいドメイン登録者が,DNSレコードを
保有している権威サーバの管理者にDNSSEC利用の申し込みを行う.
DNSSEC利用申し込みを受け入れた権威サーバ管理者は,公開鍵と秘密鍵のペア,および電子署名を作成し,署名をDNSレコードに登録する.
その後,作成した公開鍵をドメイン名登録者,ドメイン登録業務を行う指定事業者,そしてJRPSが管理するJPレジストリデータベースおよびJPドメインのルートDNSサーバへと渡す.

次いで,ユーザがDNSSECが利用されているWebページへとアクセスしようとした場合,ユーザからキャッシュDNSサーバへと問い合わせが行われる.ユーザからの問い合わせを受けたキャッシュDNSは,キャッシュを保管していない場合,通常のDNSと同様に権威サーバに対して問い合わせを行う.
権威サーバからは,電子署名の付与されたDNS応答が返される.
キャッシュサーバは,JPドメインのルートDNSサーバに保管されている公開鍵を用い,応答に付与された電子署名の正当性を検証する.
正当性が検証されたならば,ユーザに応答を返し,また問い合わせ結果をキャッシュとして自ら保管する.

DNSSECを利用することで,出自および完全性の保証が可能となる.\cite{jrps_dnssec}
DNSキャッシュポイズニングへのほぼ唯一の対抗策ということで,広く普及することが望まれてはいるが,DNSSECを実用化するためにはキャッシュDNSサーバ,権威DNSサーバ,ルートDNSサーバ,ドメイン登録者など関係者のほとんど全てがDNSSEC対応および公開鍵暗号方式への対応処理を行わねばらなず,関係者すべての協力が必要となる.そのため,導入までには少し時間がかかるのではないかと考えられる.
