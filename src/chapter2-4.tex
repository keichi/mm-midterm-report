\section{隔離の失敗}

\subsection{概要}
複数のテナント化とリソースの共有は,クラウドコンピューティングの環境を定義付ける二つの特徴であり,計算能力,ストレージおよびネットワークは複数のユーザ間で共有される.この時にストレージ,メモリ,ルーティング,および共有されるインフラストラクチャを使用する異なるテナント間での評判を隔離するメカニズムに不備が生じる可能性がある.この「隔離の失敗」が生じた際の影響としては,価値あるデータや機密データの損失,クラウドプロバイダやそのクライアントに対するサービスの中断や評判の失墜等が考えられる.

\subsection{事例とその対策}
実際に,リソースの共有を行うためには物理デバイスの接続などを行う必要があるが,この際には各デバイスの隔離の失敗などが生じる可能性がある.ここでは隔離の失敗の事例として,「サイドチャネル攻撃」と「CrossVMサイドチャネル攻撃」について触れる.

\subsubsection{サイドチャネル攻撃}
暗号解読手法の一つで,暗号を処理している装置の物理的な特性を外部から観察・測定することにより,秘密情報の取得を試みる攻撃手法.
暗号機能を内蔵したICカードなど,暗号処理機能の組み込まれた電子機器や半導体製品が主な攻撃対象で,暗号化や復号を行なうときの処理時間や消費電力の推移,外部に放出する電磁波,熱,音の変動などを継続的に測定し,入力値との相関から秘密鍵など重要な情報を割り出す手法である.
よく知られた方式として,様々な入力を与えた時の処理時間の違いを統計的に処理して内部の秘密鍵を推測する「タイミング攻撃」(timing attack),消費電力の違いから推測する「電力解析攻撃」(power analysis attack)などがある.\par 
このサイドチャネル攻撃の対策としては,「モンゴメリラダー法」\cite{sidechannel}がある.これは,電力解析によるサイドチャネル攻撃(電力解析攻撃)をベースにKocherがさらに強力な攻撃を考案した物\cite{kocher1999}に対し, その対策として注目が集まっている方法である.これは,電力の違いを生じさせずに常に同じパターンになることを目標とした方法である.  
	
\subsubsection{CrossVMサイドチャネル攻撃}
CrossVMサイドチャネル攻撃は「キャッシュ共有」,「メモリの覗き見」の大きく2つに分けることができる\cite{vmsec}.\par 
まず,「キャッシュ共有」とは,SetAssociativeCacheを共有している「悪意のあるVM」が連続してキャッシュをたたく攻撃である.キャッシュの反応が遅れると他のVMでアクセスしていることがわかってしまうため問題である.
2005年に話題になったHyperThreadingの脆弱性と同じである.\par
具体的な対策として,VMの隔離があげられる.これは,仮想サーバーが多くの点で物理サーバーと同一のセキュリティ要件があるという観点から,アプリケーションを別の仮想マシンにデプロイすることで,複数のアプリケーションを同じホスト・オペレーティング・システムで実行する場合と比較して,よりよいセキュリティ制御が提供されるという考え方である.\par 
次に,「メモリの覗き見」とは,既存の物理メモリへの覗き見攻撃のことである.メモリを冷やしての解析(ColdBootAttack)などがある.仮想マシンでは同様の攻撃がソフトウェアのみででき,管理OSをのっとられればVMIntrospection機能から可能である.VMIntrospectionとは,VMのメモリやデバイスの状態をハイパーバイザーから覗き見る技術のことであり,ゲストOS自身ではRootkitにより正しく状態を把握できない場合がある.\par
このような攻撃に対する対策としてはVMメモリの暗号化,Overshadow
などがあげられる.「Overshadow」とは,VMMがプロセスのメモリを暗号化することで,ゲストOSが乗っ取られてもメモリの機密性と一貫性を保証するシステムのことである\cite{chen2008}.VMMがゲストOS上のコンテキストを認識して,ゲストOSがプロセスのメモリアクセスする時にページを暗号化する.プロセスに対しては,暗号化されていないメモリを見せることで,プロセスは正しく動作することが可能である.メモリを暗号化した状態でもシステムコールなどOSの機能を正しく使えるように,専用のプログラムローダを用いる必要がある.\par
